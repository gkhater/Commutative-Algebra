\documentclass[11pt,a4paper]{article}

%----- ENHANCED TYPOGRAPHY -----
\usepackage[utf8]{inputenc}
\usepackage[T1]{fontenc}
\usepackage{lmodern}        % clean vector font
\usepackage{microtype}      % better justification & kerning
\usepackage{palatino} 
\usepackage{braket}    % for text & math
\usepackage{mathtools}  % in the preamble

%----- PAGE LAYOUT -----
\usepackage{geometry}
\geometry{top=1in, bottom=1in, left=1in, right=1in}
\usepackage{setspace}
\onehalfspacing  % 1.5 line spacing

%----- FANCY HEADERS & FOOTERS -----
\usepackage{fancyhdr}
\pagestyle{fancy}
\fancyhf{}
% page number outside, header text inside

\fancyhead[LO]{\small \rightmark}
\fancyhead[RO]{\small \leftmark}
\renewcommand{\headrulewidth}{0.4pt}
\renewcommand{\footrulewidth}{0pt}

\usepackage{bookmark}

% make sections feed into \leftmark/\rightmark
\renewcommand{\sectionmark}[1]{\markboth{#1}{}}
\renewcommand{\subsectionmark}[1]{\markright{#1}}

%----- SECTION NUMBERING & TOC DEPTH -----
\setcounter{secnumdepth}{3}  % number down to \subsubsection
\setcounter{tocdepth}{2}     % show ToC down to \subsection

%----- AMS MATH & THEOREM STYLES -----
\usepackage{amsmath,amssymb,mathtools}
\usepackage{amsthm}

% definitions, examples, remarks upright
\theoremstyle{definition}
\newtheorem{definition}{Definition}[section]
\newtheorem{example}[definition]{Example}
\newtheorem{remark}[definition]{Remark}

% theorems, lemmas, corollaries italic
\theoremstyle{plain}
\newtheorem{theorem}[definition]{Theorem}
\newtheorem{lemma}[definition]{Lemma}
\newtheorem{proposition}[definition]{Proposition}
\newtheorem{corollary}[definition]{Corollary}

% unnumbered proof environment
\theoremstyle{remark}

%----- OTHER PACKAGES -----
\usepackage{graphicx}
\usepackage{tikz}
\usetikzlibrary{calc, matrix, decorations.pathreplacing, positioning}
\usepackage{tikz-cd}
\usepackage{hyperref}
\hypersetup{colorlinks,
linkcolor=blue, citecolor=purple, urlcolor=teal}
\usepackage{enumitem}
\setlist[itemize]{nosep, left=1.5em}
\usepackage{booktabs}
\usepackage{listings}
\lstset{
basicstyle=\ttfamily\small,
numbers=left,
numbersep=5pt,
frame=single,
breaklines=true
}
\usepackage{xcolor}
\definecolor{shade}{HTML}{F5F5F5}
\usepackage{float}
%----- CUSTOM MACROS -----
\newcommand{\F}{\mathbb{F}}
\newcommand{\code}[1]{\texttt{#1}}
\newcommand{\dist}[2]{d\bigl(#1,#2\bigr)}
\newcommand{\R}{\mathbb{R}}
\newcommand{\Z}{\mathbb{Z}}
\newcommand{\N}{\mathbb{N}} 
\newcommand{\Q}{\mathbb{Q}} 
\newcommand{\C}{\mathbb{C}}
\renewcommand{\set}[1]{\left\{ #1 \right\}}
\newcommand{\angles}[1]{\langle #1 \rangle}
\newcommand{\abs}[1]{\lvert #1 \rvert} 
\newcommand{\norm}[1]{\lVert #1 \rVert}
\newcommand{\Bil}{\operatorname{Bil}}
\newcommand{\Hom}{\operatorname{Hom}}
\newcommand{\im}{\operatorname{Im}}
\newcommand{\coker}{\operatorname{coker}}

% \usepackage{mathtools} 
% \DeclarePairedDelimiter{\angles}{\langle}{\rangle} 
% \DeclarePairedDelimiter{\braces}{\left\{}{\right\}} 
% \DeclarePairedDelimiter{\abs}{\lvert}{\rvert} 
% \DeclarePairedDelimiter{\norm}{\lVert}{\rVert}

%----- TITLE METADATA -----
\title{\LARGE\bfseries Commutative Algebra}
\author{Georges Khater \\ \small American University of Beirut, Math 344}
\date{\today}

%===============================================
\begin{document}
\maketitle
\tableofcontents
\bigskip


\section{Introduction.}
\subsection{Multiplicative sets and Prime ideals}
\begin{definition}
    A subset $S \subset A$ is a called multiplicatively closed iff 
    \begin{enumerate}
        \item $1 \in S$ 
        \item if $s,t \in S$, then $st \in S$. 
    \end{enumerate}
\end{definition}

\begin{proposition}
    Let $S$ be multiplicatively closed, and let $a$ be an ideal with $a \cap S = \varnothing$. Consider all ideals $b$ s.t 
    $$\begin{cases}
        a \subseteq b \\ 
        b \cap S = \varnothing
    \end{cases}$$
    Equivalently $a \subset b \subset A - S$. Then the poset of such ideals satisfies the conditions of Zorn's lemma, and any maximal such element is a prime ideal of $A$.  
\end{proposition}


\begin{proof}
    The poset is nonempty because it contains $a$, every totally ordered subset $\set{b_i \colon i \in I}$, has an upper bound 
    $\bigcup_{i \in I} b_i$. Therefore by Zorn's lemma there exists some maximal element $p$ of this set. 

    Suppose that $x, y \in A$ with $xy \in p$ but $x, y \not\in p$: Since $x,y \not\in b$ then the ideal sums 
    $$p + \angles{x}, \ p + \angles{y}$$
    Strictly contain $p$, so they must contain each an element of $S$, therefore $\exists s \in S$ of the form 
    $$s = p_1 + xa_1 \quad p_1 \in p, \ a_1 \in R$$
    and similarly $\exists t \in s$ with 
    $$t = p_2 + y a_2$$
    but then $st \in S$, therefore 
    $$p_1 p_2 + p_1 y a_2 + p_2 x a_1 + xy a_1 a_2 \in P \implies st \in p$$ 
    contradicting that $p \cap S = \varnothing$. \\
    Therefore $p$ is prime.
\end{proof}

\begin{example}
    \begin{itemize}
        \item $A$ a domain and $S = A - \set{0}$. 
        \item In $\Z$: $S = \set{1, 6, 6^2, \cdots}$, or more generally in any $A$ if 
        $x \in A$ is not nilpotent, then 
        $$S = \set{1, x, x^2, \cdots}, \quad S \cap \set{0} = \varnothing$$
        There exists a prime ideal with $p \cap S = \varnothing \implies x \not\in p$ (obviously $0 \in p$). \\
        Therefore 
        $$\set{\text{all nilpotent } x} = \operatorname{nilrad} (0) = \bigcap_{p \text{ prime ideal}} p$$
        \item $p \subset A$ prime then $A - p$ is multiplicatively closed. 
    \end{itemize}
\end{example}

\begin{theorem}
    If $a \subset A$ is an ideal, then 
    $$\operatorname{rad} a := \set{x \in A \mid \ \exists N \geq 1 \text{ with } x^N \in a}$$
    satisfies 
    $$\operatorname{rad} a = \bigcap_{\text{all } p, \ p \supseteq a} p$$ 
\end{theorem}

\begin{proof}
    \begin{itemize}
        \item Slick proof: apply the conclusion above to the quotient ring $A / a$. 
        \item Pedestrian proof: Show both inclusions directly :
        If $x^N \in a$ and $a \subset p$, then $x \in p$ (for all primes $p \supseteq a$), so $x \in \bigcap_{p \supseteq a} p$. \\
        If $\forall N$, $x^N \not\in a$ then take $S = \set{1, x, x^2, \cdots}$ we have $a \cap S = \varnothing$; so $\exists p$ ($p \supseteq a$ and $p \cap S$)
        which is prime with $p \supseteq a$ and $p \cap S = \varnothing$ so $\exists p \supseteq a$ with $x \not\in p$ so 
        $$x \not\in \bigcap_{p \supseteq a} p$$ 
    \end{itemize}
\end{proof}

\subsection{The Jacobson radical of a commutative ring}
\begin{definition}
    $$\operatorname{Jrad} a = \bigcap_{\text{all maximal ideals } m \subsetneq A} \supseteq \operatorname{nilrad} A$$
\end{definition}

\begin{proposition}
    $x \in \operatorname{Jrad} A \iff \forall y \in A, \ 1 - xy \text{ is a unit in } A$. 
\end{proposition}

\begin{proof}
    \begin{itemize}
        \item[$\implies$] Let $x \in \operatorname{Jrad} A$, then $\forall m$ maximal ideal, $x \in m$. If $y \in A$ satisfies $1 - xy$ not a unit, then 
        $$\angles{1 - xy} \subsetneq A$$
        so $\exists m$ maximal with $1 - xy \in m$ but $x \in m$. This would force $1 \in m$ since 
        $$1 - xy + xy$$
        \item[$\impliedby$] Suppose that $\forall y \in A$, $1 - xy$ is a unit in $A$. We want to show that if $M$ is maximal, is $x \in m$? 
        Work in $A / m$, field: $\overline{x}$ satisfies that $\forall \overline{y} \in A / m$, 
        $$\overline{1} - \overline{xy} \text{ is a unit in } A / m$$
        but in a field, this can only happen if $\overline{x} = \overline{0}$. Therefore $x \in m$, for all $m$. 
    \end{itemize}
\end{proof}

\subsection{Extension and contraction of ideals} 
\textbf{Setup.} $A \xrightarrow{f} B$ a homomorphism of rings (so $B$ is an $A$-algebra) and 
$$\mathfrak{a} \subseteq A \text{ ideal} \to \mathfrak{a}^e = \mathfrak{a} B = \angles{f(a) \colon a \in A}$$
$$\mathfrak{b} \subseteq B \to \mathfrak{b}^c = f^{-1} (\mathfrak{b}) = \ker \left(A \xrightarrow{f} B \xrightarrow{\nu} B / \mathfrak{b}\right)$$
Basic observations 
$$\mathfrak{a}^{ec} \supseteq \mathfrak{a}, \quad \mathfrak{b}^{ce} \subseteq \mathfrak{b}$$
but notice that 
$$\mathfrak{a} = \mathfrak{a}^e, \quad \mathfrak{b}^{cec} = \mathfrak{b}^c$$
\textbf{Important fact.} if $q \subseteq B$ is prime then so is $q^c$ therefore 
$$A / q^c \xrightarrow{\text{subdomain}} B / q$$
\begin{example}
    $\Z \to \Q$, $(6 \Z)^e = \Q$, 
    $(6 \Z)^{ec} = \Z$. 
\end{example}

\begin{example}
    $\Z \to \Z / 10 \Z$
    $(4\Z)^e = \frac{4 \Z + 10 \Z}{10 \Z} = 2\Z / 10 \Z$ 
    so 
    $$(4 \Z)^{ec} = 2 \Z$$
\end{example}

\begin{example}
    $A \to A[x]$ take $\mathfrak{a} \subseteq A$ then 
    $\mathfrak{a}^e = \mathfrak{a}[x]$ which is the set of all polynomials with coefficients in $\mathfrak{a}$. And 
    $\set{0_A} = \angles{x + 1}^c$; $\angles{x+1}^{ce} = \set{0_{A[x]}}$. 
\end{example}

\subsection{Modules.}
View Math 341 Notes, I won't be taking notes here. 

Let $A$ be given and $a_1, \cdots, a_r$ ideals in $A$, we say that these ideals are relatively prime iff 
$\forall i, j$ with $i \neq j$, 
$$a_i + a_j = A \iff \exists x \in a_i, \ y \in a_j \text{ s.t } x + y = 1$$ 

\begin{proposition}
    In the above situation we have that 
    $$a_1 \cdots a_r = a_1 \cap a_2 \cap \cdots \cap a_r$$
\end{proposition}

\begin{proof}
    Induction on $r$:
    $$a_1 a_2 \subseteq a_1 \cap a_2 \quad \text{easy}$$
    Conversely if $z \in a_1 \cap a_2$ we have that $x \in a_1, \ y \in a_2$ s.t $x + y = 1$ then 
    $$z = z \cdot 1 = zx + zy \in a_1 a_2$$
    for the inductive step, we have already $b = a_1 \cap \cdots \cap a_{r-1} = a_1 \cdots a_{r-1}$ need to observe that $a_r$ is still 
    relatively prime to $b$. 

    Indeed, we know that 
    $$a_1 + a_r = A, \ \cdots, a_{r-1} + a_r = A$$
    therefore we chose elements in $x_1 \in a_1, \cdots, x_{r-1} \in a_{r-1}$ and 
    $y_1, \cdots y_{r-1}\in a_r$ with $x_i + y_i = 1$. We can take 
    \begin{align*}
        1 &= 1^{r-1}\\ 
        &= (x_1 + y_1) (x_2 + y_2) \cdots (x_{r-1} + y_{r-1})\\
        &= x_1 \cdots x_{r-1} + (\text{terms involving some $y_i$}) \\
        &= \in a_{1} \cdots a_{r-1} + \in a_r 
    \end{align*}
\end{proof}

\begin{theorem}[Chinese Remainder Theorem]
    Let $a_1 \cdots a_r$ be pairwise relatively prime, consider the homomorphism of $A$ modules 
    $$A \xrightarrow{f} A / a_1 \oplus \cdots \oplus A / a_r$$
    Hence 
    $$A / a_1 \oplus \cdots \oplus A / a_r \cong A / \ker = A / a_1 \cdots a_r$$
\end{theorem}

\subsection{Local Rings}
\begin{definition}
    A ring $A$ is said to be \emph{local} iff it has exactly one maximal ideal (called $m$).
\end{definition} 

\begin{example}
    \begin{itemize}
        \item $A = K$ a field, then $m = 0$ 
        \item Fix $p$ prime $A \set{a/b \in \Q \mid a, b \in \Z, \ b \not\in p \Z} \supset m = \set{pc / b \mid c,b \in \Z, \ p \not\mid b} = p A$. 
        Check that the only ideals are 
        $$A \supsetneq pA \supsetneq p^2 A \supsetneq \cdots$$
        \item \begin{align*}
            A &= k [[x_1,x_2]] \\
            &= \set{\text{formal power series}} \\
            &= \set{a_0 + b_1 x_1 + b_2 x_2 + c_{11} x_1 + \cdots} 
        \end{align*}
        Where $a_i, b_j, c_{k.l}, \cdots \in K$ with 
        $$m = \angles{x_1, x_2} = \set{\text{formal power series with } a_0 = 0}$$
        This is the only maximal ideal because $m$ is the set of non-units, think about it. 
        Same for the example above.
    \end{itemize}
\end{example}

\begin{proposition}
    $A$ is local $\iff$ the set of non-units of $A$ is an ideal (which turns out to be the unique maximal ideal).
\end{proposition}

\begin{proof}
    We denote by $V$ the set of non-units of $A$
    \begin{itemize}
        \item[$\implies$] Let $A$ be local, let $m$ be its unique maximal ideal we clearly have that $m \subseteq V$
        Now take any $v \in V$, then by Zorn's lemma we can find some maximal ideal $I \ni a$. But since $m$ is the unique maximal ideal then we have 
        that $m \ni a$. 

        \item[$\impliedby$] Trivial.
    \end{itemize}
\end{proof}

\subsection{Algebras} 
\begin{definition}
    An \emph{$A$-algebra} is a ring $R$ (not necc. commutative) with a ring homomorphism $f \colon A \to R$ s.t 
    $$\operatorname{Im} f \subseteq \text{ center of } R$$
    We will generally stick to the case where $R$ is commutative. 
\end{definition}

\begin{example}
    All rings are $\Z$-algebras, $M_n(A)$, $A[x_1, \cdots, x_r]$. Moreover $\C$, $\mathbb{H}$ and matrices in $\C$ are all $\R$ algebras
\end{example}

\begin{definition}
    A \emph{finitely generated $A$-algebra} means that $\exists a \subseteq A$ an ideal s.t 
    $$R \cong A[x_1, \cdots,x_n] / a$$
\end{definition}

\begin{example}
    A $\Z$-algebra $\Z / 10 \Z [y_1, y_2]$ where $y_1^2 - 1 = 0$ and $y_2^3 - 2 y_1 y_2 - 3$ 
    which is isomorphic to 
    $$\Z [x_1, x_2] / \angles{10, x_1^2 - 1, x_2^3 - 2x, x_2 - 3}$$
\end{example}

\begin{definition}
    $a, b$ ideals of $A$ then the \emph{colon ideal} $(a \colon b)$ is defined to be 
    $$(a \colon b) = \set{x \in A \colon x b \subseteq a} = \bigcap_{y \in b} \operatorname{Ann} (y + a \in A / a)$$
\end{definition}

\begin{example}
    In $\Z$, 
    $$(4 \Z \colon 10 \Z) = 2 \Z, \quad (10 \Z \colon 4 \Z) = 5 \Z$$
\end{example}

\begin{theorem}
    Let $p_1, \cdots, p_r$ be prime ideals of $A$ and suppose that $a$ satisfies 
    $$a \subseteq p_1 \cup \cdots \cup p_n$$
    Then $\exists j$ s.t $a \subseteq p_j$.

    (Note that the proof below also works if all the $p_i$'s are prime except except for $1$ or $2$ $p_i$'s).
\end{theorem}

\begin{proof}
    By induction on $n$: 
    \begin{itemize}
        \item $n = 1$: trivial. 
        \item $n = 2$: Suppose $a \subseteq p_1 \cup p_2$ but $a \not\subseteq p_1$ and $a \not\subseteq p_2$ so $\exists x_1, x_2 \in a$ s.t $x_1 \not\in p_1$, 
        $x_2 \not\in p_2$. Therefore we must have $x_1 \in p_2$ and $x_2 \in p_1$, then 
        $$a \ni z := x_1 + x_2 \not\in p_1 \cup p_2$$
        (If $z \in p_1$ then $x_1 = z - x_2 \in p_1$). 
        \item Now suppose $n \geq 3$ and we know the result for $n-1$ primes. Suppose that 
        $$a \subseteq p_1 \cup \cdots \cup p_n$$
        if $a \subseteq p_1 \cup \cdots p_{n-1}$ (after potentially reordering), then we are okay. 
        So suppose that $a \not\subseteq p_1 \cup \cdots \cup p_{n-1}$ for any reordering. so $\exists x_j \in a$ with 
        with $x_j \in p_j$ for every $j$. 
        
        Consider $z = x_1 + x_2 x_3 \cdots x_n$, of course $z \in a$ but $z \not\in p_j$ for any $j$ (contradiction)!
        
        Now let $x_1 \in p_1$ and $x_2, \cdots, x_n \not\in p_1$ and $p_1$ is prime therefore 
        $$x_2 \cdots x_n \not\in p_1$$
        hence $z = x_1 + (x_2 \cdots x_n) \not\in p_1$. Now for $p_2$, we have that 
        $$x_1 \not\in p_2, \ x_2 \in p_2 \implies x_1 + x_2 x_3 \cdots x_n \not\in p_2$$
        Similarly for the other $p$'s. 
    \end{itemize}
\end{proof}

\section{Modules}
\begin{lemma}[Nakayama's lemma]
    Let $A$ be a local ring, with maximal ideal $m$; and Let $M$ be a finitely generated $A$-module. Suppose that 
    $$m M = M$$
    (where $m M$ is the set of all finite sums $m_1 y_1 + \cdots + m_r y_r$ with $m_i \in m$ and $y_i \in M$.)\\
    Then $M = 0$.      
\end{lemma}

\begin{proof}[Cheap proof]
    Suppose $M = \angles{x_1, \cdots, x_r}$, we can remove redundant $x_i$'s, therefore we can assume that 
    $$\forall j, \ x_j \in \angles{x_1, \cdots, \hat{j}, \cdots, x_r}$$
    If $M \neq 0$ we are left with this set of generators with $r \ge 1$. 
    But then $x_1 \in M = m M$ therefore 
    $$x_1 = \sum_{i=1}^r m_i x_i \quad m_i \in m$$
    Therefore $(1- m_1) x_1 = m_2 z_2 + \cdots + m_r x_r$ but $1 - m_1 \not\in m$ must be a unit in $A$! Therefore 
    $$x_1 \in \angles{x_2,\cdots, x_r} \quad \text{impossible}$$     
\end{proof}

\begin{corollary}
    Fix $(A, m)$ a local ring. Suppose that $M \supseteq N$, $M$ finitely generated and $M = N + m M$ (Equivalently in $M / m M$ the images of $N$ generate everything). 
    Then $M = N$. 
\end{corollary}

\begin{proof}
    Apply Nakayama to $M / N$; note that 
    $$m \cdot (M/N) = M/N, \quad M/N \text{ is finitely generated}$$
\end{proof}

\begin{theorem}[General Statement of Nakayama's Lemma]
    Let $A$ be any (commutative!) ring and let $M$ be a finitely generated $A$-module, let $a \subseteq A$ be an ideal (usually $a = \operatorname{Jrad} A$) and suppose 
    $$a M = M$$
    then $\exists b \in 1 + a$ s.t $b M = 0$. 

    In the case where $a$ is the $\operatorname{Jrad} A$ in which case $b \in 1 + a$ is a unit, so $M = 0$. 
\end{theorem}

\begin{proof}
    Let $M = \angles{x_1, \cdots, x_r}$, since $M = aM$, we see that $\exists a_{ij} \in a$ s.t 
    \begin{align*}
        x_1 &= a_{11} x_1 + \cdots + a_{1r} x_r \in a M \\
        x_2 &= a_{21} x_1 + \cdots + a_{2r} x_r \in a_M \\
        &\vdots \\
        x_r &= a_{r1} x_1 + \cdots + a_{rr} x_r\in a_M
    \end{align*}
    Let $P = (a_{ij})_{ij}$ be the matrix of $a_{ij}$'s, rewrite the system above as 
    \begin{align*}
        (1 - a_{11}) x_1 - a_{12} x_2 + \cdots - a_{1r} x_r &= 0 \\
        a_{21} x_1 + (1 - a_{22}) x_2 + \cdots - a_{2r} x_r &= 0 \\
        &\vdots \\
    \end{align*}
    Symbolically we get that 
    $$(I - P) \begin{pmatrix}
        x_1 \\ x_2 \\ \vdots \\ x_r
    \end{pmatrix} = \begin{pmatrix}
        0 \\ 0 \\ \vdots \\0
    \end{pmatrix}$$
    Multiplying on the left by the matrix $Q = (I - P)^{adj} \in M_r(A)$; it has the property that 
    $$Q (I-P) = \operatorname{diag} (b)$$
    where $b = \operatorname{det} (I - P) \in 1 + a$; we deduce 
    $$b x_1 = 0, \ b x_2 = 0, \cdots, \ b x_r = 0$$
    so $b$ annihilates every generator of $M$, and therefore annihilates all of $M$. 
\end{proof}

\begin{example}
    We can use this so that if $(A, m)$ then 
    $$m \supseteq m^2 \supseteq m^3 \supseteq \cdots$$
    if $m = m^{n+1}$ then $m^n = 0$. Indeed $m_i / m_{i+1}$ is an $A / m$-module.
\end{example}

\subsection{Hom functors and exactness} 

\paragraph{Quick reminder.}
$M, N$ are $A$ Modules, therefore $\operatorname{Hom}_A (M, N)$ is an $A$-module (since $A$ is commutative). Some facts: 
\begin{itemize}
    \item We have a natural isomorphism $\operatorname{Hom} (A, N) \cong N$ (as $A$-modules). 
    \item Have an natural isomorphism 
    $$\operatorname{Hom} \left( \bigoplus_{\alpha \in I} M_\alpha, N\right) \cong \prod_{\alpha \in I} \operatorname{Hom} (M_\alpha, N)$$
    $$(\cdots x_\alpha \cdots)_{\text{almost all zeroes}} \xrightarrow{\varphi} \sum_\alpha \varphi_\alpha (x_\alpha)$$
    But also 
    $$\operatorname{Hom} (M, \prod_{\beta} N_\beta) \cong \prod_\beta \operatorname{Hom} (M, N_\beta)$$
    $$x \in M \to (\cdots \psi_\beta (x) \cdots ) \in \prod_\beta N_\beta$$

    \item $0 \to N' \xrightarrow{f} N \xrightarrow{g} N''$ exact implies that 
    $$0 \to \operatorname{Hom} (M, N') \xrightarrow{f_*} \operatorname{Hom} (M, N) \xrightarrow{g_*} \operatorname{Hom} (M, N'')$$
    (even if $g$ is surjective, then $g_*$ need not be surjective). 

    Similarly if 
    $$M' \xrightarrow{u} M \xrightarrow{w} M'' \to 0 \quad \text{exact}$$
    then 
    $$0 \to \operatorname{Hom} (M'', N) \xrightarrow{w^*} \operatorname{Hom} (M, N) \xrightarrow{u^*} \operatorname{Hom} (M', N)$$
    (even if $U$ is injective then $U^*$ need not be surjective).
\end{itemize}
\subsection{Presentations of Modules}

Let $A$ be a (commutative) ring and $M$ an $A$-module.

\paragraph{Finite presentation.}
Say $M$ is generated by $x_1,\dots,x_n$ with relations $r_1,\dots,r_k$, where
\[
r_j:\quad a_{j1}x_1+\cdots+a_{jn}x_n=0 \qquad (1\le j\le k).
\]
Let $A^n$ have basis $e_1,\dots,e_n$ and $A^k$ have basis $f_1,\dots,f_k$.
Define $\eta:A^n\to M$ by $\eta(e_i)=x_i$ and define $\Phi:A^k\to A^n$ by
\[
\Phi(f_j)=a_{j1}e_1+\cdots+a_{jn}e_n.
\]
Then $\eta\circ \Phi=0$, and $\im(\Phi)=\ker(\eta)$ (the relations generate all relations), hence we have an exact sequence
\[
A^k \xrightarrow{\ \Phi\ } A^n \xrightarrow{\ \eta\ } M \to 0,
\qquad\text{so } M \cong \coker(\Phi)=A^n/\im(\Phi).
\]
If we write $\Phi$ as a matrix, it is the $n\times k$ matrix $(a_{ji})$ with the convention that the $j$-th column is the coordinate vector of $\Phi(f_j)$ in the basis $(e_i)$.

\paragraph{Infinite presentations.}
Nothing changes: if $M$ is generated by a family $(x_i)_{i\in I}$ with relations indexed by $J$, one has
\[
A^{(J)} \xrightarrow{\ \Phi\ } A^{(I)} \xrightarrow{\ \eta\ } M \to 0,
\]
where $A^{(I)}=\bigoplus_{i\in I} A$ and $A^{(J)}=\bigoplus_{j\in J} A$ are free modules on those bases.

\subsection*{Computing \texorpdfstring{$\Hom_A(M,N)$}{Hom(M,N)} from a presentation}

Given an exact sequence
\[
A^k \xrightarrow{\ \Phi\ } A^n \xrightarrow{\ \eta\ } M \to 0,
\]
apply the contravariant left-exact functor $\Hom_A(-,N)$ to obtain the exact sequence
\[
0 \to \Hom_A(M,N) \xrightarrow{\ \eta^* \ } \Hom_A(A^n,N) \xrightarrow{\ \Phi^* \ } \Hom_A(A^k,N),
\]
where $\eta^*(\psi)=\psi\circ \eta$ and $\Phi^*(\varphi)=\varphi\circ \Phi$.

\paragraph{Identifications.}
\begin{itemize}
\item If $n,k<\infty$ then $\Hom_A(A^n,N)\cong N^n$ and $\Hom_A(A^k,N)\cong N^k$ via
\[
\Hom_A(A^n,N)\ni \psi \longleftrightarrow (\psi(e_1),\dots,\psi(e_n))\in N^n.
\]
\item More generally, for a free module $A^{(I)}=\bigoplus_{i\in I} A$, one has
\[
\Hom_A\!\Big(\bigoplus_{i\in I} A,\,N\Big)\cong \prod_{i\in I} \Hom_A(A,N)\cong \prod_{i\in I} N.
\]
(Important: \emph{direct sum} on the left becomes a \emph{direct product} on the right.)
\end{itemize}

Under the finite identifications, the map $\Phi^*:N^n\to N^k$ is given explicitly by
\[
(y_1,\dots,y_n)\longmapsto (z_1,\dots,z_k),
\qquad
z_j=a_{j1}y_1+\cdots+a_{jn}y_n.
\]
Therefore
\[
\Hom_A(M,N)\cong \ker\!\big(N^n \xrightarrow{\ \Phi^*\ } N^k\big).
\]
Equivalently: an $A$-linear map $M\to N$ is determined by the images of the generators $x_i$, and the only constraints are that the relations $r_j$ hold after applying the map.

% --------------------------------------------------------------------

\subsection{Tensor Product of Modules (review)}

Assume $A$ is commutative (so that left/right issues disappear).

\paragraph{Bilinear maps.}
For $A$-modules $M,N,P$, let $\Bil_A(M\times N,P)$ be the set of maps
$\beta:M\times N\to P$ which are $A$-linear in each variable:
\[
\beta(x_1+x_2,y)=\beta(x_1,y)+\beta(x_2,y),\quad
\beta(ax,y)=a\beta(x,y),
\]
and similarly in the second variable.

There is a natural bijection
\[
\Bil_A(M\times N,P)\ \cong_{\mathrm{nat}}\ \Hom_A\big(M,\Hom_A(N,P)\big),
\]
sending $\beta$ to the map $x\mapsto (y\mapsto \beta(x,y))$.

\paragraph{Universal property of the tensor product.}
The tensor product $M\otimes_A N$ is an $A$-module equipped with a bilinear map
\[
\tau:M\times N\to M\otimes_A N,\qquad (x,y)\mapsto x\otimes y,
\]
such that for every $A$-module $P$, composition with $\tau$ induces a natural isomorphism
\[
\Hom_A(M\otimes_A N,\,P)\ \cong_{\mathrm{nat}}\ \Bil_A(M\times N,\,P).
\]

\paragraph{Concrete construction.}
One can construct $M\otimes_A N$ as the quotient of the free $A$-module
$F$ on symbols $(x,y)$, by the submodule generated by the bilinearity relations:
\begin{align*}
(x_1+x_2,y)-(x_1,y)-(x_2,y),\qquad &(ax,y)-a(x,y),\\
(x,y_1+y_2)-(x,y_1)-(x,y_2),\qquad &(x,ay)-a(x,y).
\end{align*}
The class of $(x,y)$ is denoted $x\otimes y$.

\paragraph{Standard isomorphisms.}
There are natural isomorphisms
\[
A\otimes_A N \cong N,\qquad
\Big(\bigoplus_\alpha M_\alpha\Big)\otimes_A N \cong \bigoplus_\alpha (M_\alpha\otimes_A N),
\qquad
M\otimes_A N \cong N\otimes_A M,
\]
and associativity up to canonical isomorphism:
\[
(M\otimes_A N)\otimes_A Z \cong M\otimes_A (N\otimes_A Z).
\]

\subsection*{Functoriality of \texorpdfstring{$\otimes$}{tensor}}

Given $A$-linear maps $f:M_1\to M_2$ and $g:N_1\to N_2$, there is a unique $A$-linear map
\[
f\otimes g:\ M_1\otimes_A N_1 \to M_2\otimes_A N_2
\]
satisfying on simple tensors
\[
(f\otimes g)(x\otimes y)=f(x)\otimes g(y).
\]
Hence for a general tensor $t=\sum_i a_i(x_i\otimes y_i)$,
\[
(f\otimes g)(t)=\sum_i a_i\big(f(x_i)\otimes g(y_i)\big).
\]

\subsection*{Extension of scalars}

Let $f:A\to B$ be a ring homomorphism and let $M$ be an $A$-module. Then $B\otimes_A M$
is naturally a \emph{$B$-module} via multiplication on the left factor:
\[
b\cdot (b'\otimes m) := (bb')\otimes m.
\]
(Equivalently: the action comes from $\mu_b\otimes 1_M$, where $\mu_b:B\to B$ is multiplication by $b$.)

\begin{example}[Polynomials]
If $B=A[x]$, then as an $A$-module,
\[
A[x]\cong \bigoplus_{d\ge 0} A\cdot x^d,
\]
so
\[
A[x]\otimes_A M \cong \bigoplus_{d\ge 0} (A\cdot x^d)\otimes_A M \cong \bigoplus_{d\ge 0} M.
\]
Under this identification, $x^d\otimes m$ corresponds to $m\,x^d$, and one gets
\[
A[x]\otimes_A M \cong M[x]
\]
(the module of polynomials with coefficients in $M$).

If $\mathfrak a\subseteq A$ is an ideal (viewed as an $A$-module), then
\[
A[x]\otimes_A \mathfrak a \cong \mathfrak a[x].
\]
Inside $A[x]$, this is the extended ideal $\mathfrak a^e := \mathfrak a\cdot A[x]$.
\end{example}

\paragraph{Tensoring algebras.}
If $A$ is commutative and $B,C$ are (commutative) $A$-algebras, then $B\otimes_A C$ is an $A$-algebra with multiplication determined by
\[
(b\otimes c)\cdot(b'\otimes c') := (bb')\otimes (cc').
\]

\begin{example}
There is a natural isomorphism of $A$-algebras
\[
A[x]\otimes_A B \cong B[x].
\]
\end{example}

% --------------------------------------------------------------------

\subsection{Exactness properties of \texorpdfstring{$-\otimes_A N$}{- tensor N}}

\paragraph{Right exactness.}
The functor $-\otimes_A N$ is \emph{right exact}: if
\[
M' \xrightarrow{f} M \xrightarrow{g} M'' \to 0
\]
is exact, then
\[
M'\otimes_A N \xrightarrow{f\otimes 1} M\otimes_A N \xrightarrow{g\otimes 1} M''\otimes_A N \to 0
\]
is exact.

Warning: it need not be left exact (injectivity can fail).

\paragraph{Example: quotients / extended ideals.}
From $0\to \mathfrak a \to A \to A/\mathfrak a \to 0$ we get
\[
\mathfrak a\otimes_A B \to A\otimes_A B \to (A/\mathfrak a)\otimes_A B \to 0.
\]
Using $A\otimes_A B\cong B$, the last map identifies $(A/\mathfrak a)\otimes_A B$ with
\[
B/\mathfrak a^e,\qquad \mathfrak a^e=\mathfrak a\cdot B.
\]

\subsection*{Computing \texorpdfstring{$M\otimes_A N$}{M tensor N} from a presentation}

If $M$ has a presentation
\[
A^k \xrightarrow{\ \Phi\ } A^n \to M \to 0,
\]
then tensoring with $N$ gives an exact sequence
\[
N^k \cong A^k\otimes_A N \xrightarrow{\ \Phi\otimes 1\ } A^n\otimes_A N \cong N^n \to M\otimes_A N \to 0.
\]
Hence
\[
M\otimes_A N \cong \coker(\Phi\otimes 1),
\]
and under the identifications $A^k\otimes N\cong N^k$, $A^n\otimes N\cong N^n$, the map
$\Phi\otimes 1: N^k\to N^n$ is given by the \emph{same matrix} $(a_{ji})$:
\[
(u_1,\dots,u_k)\longmapsto \Big(\sum_{j=1}^k a_{j1}u_j,\ \dots,\ \sum_{j=1}^k a_{jn}u_j\Big).
\]

Equivalently: $M\otimes_A N$ is generated by $x_i\otimes y$ ($i=1,\dots,n$, $y\in N$),
with relations coming from:
\begin{itemize}
\item bilinearity in $y$ (so $x_i\otimes (y+y')=x_i\otimes y+x_i\otimes y'$ and $x_i\otimes (ay)=a(x_i\otimes y)$),
\item and the original relations of $M$: if $\sum_i c_i x_i=0$ in $M$, then for every $y\in N$,
\[
\sum_i c_i (x_i\otimes y)=0 \quad \text{in } M\otimes_A N.
\]
\end{itemize}

\subsection*{Conceptual proof of right exactness (via adjunction)}

\subsection*{Conceptual proof of right exactness (via adjunction)}

Recall the adjunction: for any $A$-modules $X,N,P$ there is a natural isomorphism
\[
\Hom_A(X\otimes_A N,\,P)\ \cong\ \Hom_A\big(X,\Hom_A(N,P)\big),
\]
equivalently
\[
\Hom_A(X\otimes_A N,\,P)\ \cong\ \Bil_A(X\times N,\,P).
\]

\begin{lemma}
Let $Q \xrightarrow{\lambda} R \xrightarrow{\mu} S \to 0$ be a sequence of $A$-modules.
Assume that for every $A$-module $P$ the sequence
\[
0 \to \Hom_A(S,P) \xrightarrow{\mu^*} \Hom_A(R,P) \xrightarrow{\lambda^*} \Hom_A(Q,P)
\]
is exact. Then $Q \xrightarrow{\lambda} R \xrightarrow{\mu} S \to 0$ is exact.
\end{lemma}

\begin{proof}[Sketch]
Surjectivity of $\mu$ follows by taking $P=S/\im(\mu)$ and considering the natural quotient map.
To show $\ker(\mu)\subseteq \im(\lambda)$, take $P=\coker(\lambda)=R/\im(\lambda)$ and let
$\varphi:R\to \coker(\lambda)$ be the quotient map. Then $\varphi\circ\lambda=0$, so
$\varphi\in\ker(\lambda^*)=\im(\mu^*)$, hence $\varphi=\psi\circ\mu$ for some $\psi:S\to\coker(\lambda)$.
If $r\in\ker(\mu)$ then $\varphi(r)=\psi(\mu(r))=0$, so $r\in\ker(\varphi)=\im(\lambda)$.
\end{proof}

Now suppose $M' \xrightarrow{f} M \xrightarrow{g} M'' \to 0$ is exact.
Fix an $A$-module $P$ and apply the left exact functor
$\Hom_A(-,\Hom_A(N,P))$ to obtain an exact sequence
\[
0\to \Hom_A(M'',\Hom_A(N,P))
\to \Hom_A(M,\Hom_A(N,P))
\to \Hom_A(M',\Hom_A(N,P)).
\]
Using the adjunction isomorphisms, this becomes
\[
0\to \Hom_A(M''\otimes_A N,P)
\to \Hom_A(M\otimes_A N,P)
\to \Hom_A(M'\otimes_A N,P).
\]
Since this holds for all $P$, the lemma implies that
\[
M'\otimes_A N \xrightarrow{f\otimes 1} M\otimes_A N \xrightarrow{g\otimes 1} M''\otimes_A N \to 0
\]
is exact, i.e. $-\otimes_A N$ is right exact.

\section{Localization of rings and modules} 

\textbf{Intuition for example.}
Let $A$ be a ring (pretend $A = k[x_1, \cdots, x_n] / \mathfrak{a}$ = polynomial functions on $V(\mathfrak{a}) \subseteq k^n$).
So $f \in A$ is basically $f \colon V(\mathfrak{a}) \to k$, let $V(f) = V(\angles{f} + \mathfrak{a})$ be the set where $f = 0$.
We can look at 
$$V(\mathfrak{a}) - V(f).$$
What are polynomial functions on this? Intuitively they look like fractions
$$a / f^n,\qquad a\in A,\ n\ge 0,$$
i.e.\ we allow denominators which are powers of $f$ (so we ``ignore'' what happens on $V(f)$).

We try to define 
$$A_f = \set{\frac{a}{f^n} \colon a \in A, \ n \ge 0}.$$
Question: when is $\frac{a}{f^n} = \frac{b}{f^m}$?
Note we can rewrite as
$$\frac{a f^{m+k}}{f^{n+m+k}} = \frac{b f^{n + k}}{f^{n+m+k}}.$$
Therefore $\frac{a}{f^n} = \frac{b}{f^m}$ iff $\exists k$ s.t 
$$f^k(af^m) = f^k(bf^n),$$
i.e.
$$f^k(af^m - bf^n) = 0.$$

\begin{remark}
    In our hw look at 2.1, this describes exactly $A[x] / \angles{xf-1}$.
\end{remark}

\begin{definition}
    Let $A$ be a ring, let $S \subseteq A$ be a multiplicatively closed set, define a relation $\sim$ on $A \times S$ as follows: 
    $$(a, s) \sim (b,t) \iff \exists u \in S \text{ s.t } u(ta - sb)=0.$$
\end{definition}

\begin{example}
    if $f \in A$, the set 
    $$\set{1, f, f^2, \cdots}$$
    is multiplicatively closed. 

    If $\mathfrak{p} \subset A$ is prime, then $A - \mathfrak{p}$ is multiplicatively closed. 
\end{example}

\begin{proposition}
    $\sim$ is an equivalence relation. 
\end{proposition}

\begin{proof}
    Exercise.
\end{proof}

\begin{definition}
    We define $S^{-1}A$ (also written as $A_S$) to be the set of equivalence classes of $\sim$. So an element of $S^{-1} A$ is a symbol 
    $$\frac{a}{s}.$$
    Note that 
    $$\frac{a}{s} = \frac{b}{t} \iff \exists u \in S \text{ s.t } u(ta - sb)=0.$$
\end{definition}

\begin{proposition}
    The operations 
    \begin{align*}
        \frac{a}{s} + \frac{b}{t} &= \frac{ta + sb}{st} \\
        \frac{a}{s} \cdot \frac{b}{t} &= \frac{ab}{st}
    \end{align*}    
    are well defined, and they turn $S^{-1}A$ into a ring with 
    $$0 = \frac{0}{1}, \ 1 = \frac{1}{1}.$$
\end{proposition}

\textbf{Notation.}
If $S = \set{1,f,f^2, \cdots}$ then we write $A_f$ instead of $S^{-1}A$.
If $S = A - \mathfrak{p}$ we write $A_{\mathfrak{p}}$.

\begin{example}
    In $\Z$ we write 
    $$\Z_f = \set{\frac{a}{f^n} \in \Q \mid \ a \in \Z, \ n \ge 0}.$$
    And 
    $$\Z_{\angles{p}} = \set{\frac{a}{b} \in \Q \mid \ a,b \in \Z,\ p \nmid b}.$$
\end{example}

\begin{remark}
    There is a ring homomorphism $\varphi \colon A \to S^{-1} A$ by $\varphi(a) = a/1$. But note that $\varphi$ is not necessarily injective.

    For example take $(\Z / 6 \Z)$ localized at $S = \set{1,3}$. Then 
    $$S^{-1}(\Z / 6 \Z) = \set{0/1, 1/1} \cong \Z / 2 \Z.$$
    This is because $3$ becomes a unit, so the $\Z/3\Z$-part disappears. Concretely:
    \begin{align*}
        2/1 &= \frac{3 \cdot 2}{3} = 0,
    \end{align*}
    hence $\set{0,2,4} \subseteq \ker \varphi$. Also
    $$\frac{1}{3} = \frac{3 \cdot 1}{3\cdot 3} = \frac{3}{3} = \frac{1}{1}.$$
    Using CRT: $\Z/6\Z \cong \Z/2\Z \times \Z/3\Z$. Inverting $3$ makes $(1,0)$ invertible, which forces the second factor to be $0$, leaving $\Z/2\Z$.
\end{remark}

\begin{example}
    Let $A=\C[x,y]/\angles{xy}$, these are functions on the union of the two coordinate axes in $\C^2$. Localize at $x$:
    $$A_x \cong \C[x,x^{-1}].$$
    Therefore 
    $$\frac{y}{1} = \frac{xy}{x} = 0,$$
    so $y \in \ker \varphi$ (intuitively: on $D(x)$ we are away from the $y$-axis, so $y$ vanishes there).
\end{example}

We also have the UMP. Let $\iota:A\to S^{-1}A$ be $\iota(a)=a/1$. Then for any ring $B$ and any ring map $g:A\to B$ such that $g(S)\subseteq B^\times$, there exists a unique ring map $\tilde g:S^{-1}A\to B$ with $\tilde g\circ \iota=g$. Concretely
$$\tilde g(a/s)=g(a)\,g(s)^{-1}.$$
The diagram is:
\[
\begin{tikzcd}
A \arrow[r,"\iota"] \arrow[dr,"g"'] & S^{-1}A \arrow[d,dashed,"\tilde g"] \\
& B
\end{tikzcd}
\]

\begin{remark}
    $0 \in S$ $\iff S^{-1} A$ is the zero ring.  
\end{remark}

Soon we will also localize modules; we will get a module over $S^{-1}A$.
Define
$$S^{-1} M = \set{\frac{x}{s} \mid x \in M, \ s \in S}$$
with 
$$\frac{x}{s} = \frac{y}{t} \iff \exists u \in S \text{ s.t } u(tx - sy)=0.$$

\subsection{Ideals in the Localization} 
We will especially look at extensions and contractions of ideals for the homomorphism
$$\varphi \colon A \to S^{-1} A.$$

\begin{proposition}
    If $\mathfrak{b} \subseteq S^{-1}A$ is an ideal, then 
    $$\mathfrak{b}^{ce} = \mathfrak{b}.$$
    Consequently all ideals of a localization are extensions of something. 
\end{proposition}

\begin{proof}
    $\mathfrak{b} \supseteq \mathfrak{b}^{ce}$ (always). 

    Conversely, let $\frac{a}{s} \in \mathfrak{b}$. Then $s (\frac{a}{s}) = \frac{a}{1}$ so $a \in \mathfrak{b}^c$, hence $\frac{a}{1} \in \mathfrak{b}^{ce}$ and so
    $$\frac{1}{s} \cdot \frac{a}{1} = \frac{a}{s} \in \mathfrak{b}^{ce}.$$
\end{proof}

\begin{proposition}
    Let $\mathfrak{a} \subseteq A$ be an ideal. Then 
    $$\mathfrak{a}^e = \set{\frac{a}{s} \mid a \in \mathfrak{a}, \ s \in S}$$
    (i.e.\ $S^{-1}\mathfrak a$ inside $S^{-1}A$), and
    $$\mathfrak{a}^{ec} = \set{x \in A \mid \exists s \in S \text{ s.t } sx \in \mathfrak{a}}.$$
\end{proposition}

\begin{example}
    Think of $A = \Z$ and $S = \set{1, 6, 6^2, \cdots}$ therefore 
    $$S^{-1} A = \Z[1/6].$$
    Then $(99 \Z)^{ec} = 11 \Z = (11 \Z)^{ec}$.
    Indeed $x\in(99\Z)^{ec}$ iff $\exists n$ such that $6^n x\in 99\Z$.
    For $n\ge 2$, $6^n$ contributes a factor $3^n$, so the $3^2$ part is automatic, and the condition becomes exactly $11\mid x$.
\end{example}

Notice that elements of $\mathfrak{a}^e$ are of the form 
$$\varphi(a_1) \cdot (\frac{b_1}{s_1}) + \cdots + \varphi(a_k) \cdot (\frac{b_k}{s_k})$$
with $a_i \in \mathfrak{a}, \ b_i \in A, \ s_i \in S$. Take a common denominator 
$$s = s_1 \cdots s_k,$$
so $\frac{b_i}{s_i} = \frac{c_i}{s}$ with $c_i = b_i \prod_{j\ne i} s_j$, hence wlog we have 
$$\frac{a_1}{1} \cdot \frac{c_1}{s} + \cdots + \frac{a_k}{1} \cdot \frac{c_k}{s}.$$

\begin{corollary}
    If $\mathfrak{a} = \angles{x_1, \cdots, x_r} \subset A$ then 
    $$\mathfrak{a}^e = \angles{\frac{x_1}{1}, \cdots, \frac{x_r}{1}}.$$
\end{corollary}

\begin{corollary}
    If $A$ is Noetherian, then so is $S^{-1} A$ (and every ideal of $S^{-1}A$) by the above corollary.
\end{corollary}

\paragraph{Prime ideals in $S^{-1}A$}
\begin{theorem}
    \begin{enumerate}
        \item 
        If $\mathfrak{q} \subset S^{-1}A$ is prime, then 
        $\mathfrak{p} := \mathfrak{q}^c \subset A$ is prime and 
        $$\mathfrak{p} \cap S = \varnothing.$$

        \item If $\mathfrak{p} \subset A$ is prime and 
        $$\mathfrak{p} \cap S = \varnothing,$$
        then $\mathfrak{p}^e \subset S^{-1}A$ is also prime. 

        \item Extension and contraction are mutually inverse bijections between the sets 
        $$\set{\mathfrak{p} \subset A \text{ prime} \mid \mathfrak{p} \cap S = \varnothing} \leftrightarrows \set{\mathfrak{q} \subset S^{-1}A \ \text{prime}}.$$
    \end{enumerate}
\end{theorem}

\begin{remark}
    This clarifies: if $\mathfrak{a} \subset A$ and $\mathfrak{a} \cap S = \varnothing$, then $\mathfrak{a}^e \subsetneq S^{-1}A$.
    Hence $\mathfrak{a}^e$ is contained in some maximal ideal $\mathfrak{m}\subset S^{-1}A$.
    Contracting gives a prime $\mathfrak{p}=\mathfrak{m}^c\subset A$ with $\mathfrak{a}\subseteq \mathfrak{p}$ and $\mathfrak{p}\cap S=\varnothing$.
\end{remark}

\begin{proof}
    \begin{enumerate}
        \item We are given $\mathfrak{q}$ is prime, then $\mathfrak{q}^c$ is prime (contraction preserves primality).
        Also if $\exists s \in S \cap \mathfrak{q}^c$ then $s/1 \in \mathfrak{q}$, but $s/1$ is a unit in $S^{-1}A$ (inverse $1/s$), contradiction since $\mathfrak{q} \subsetneq S^{-1}A$. 

        \item Suppose we have two elements of $S^{-1}A$ whose product is in $\mathfrak{p}^e$.
        Call these elements $a/s$ and $b/t$ with $a,b\in A$ and $s,t\in S$.
        Then 
        $$\frac{ab}{st} \in \mathfrak{p}^e.$$
        So $\exists u \in S$ such that $uab \in \mathfrak{p}$ (clearing denominators).
        Since $u\in S$ and $\mathfrak{p}\cap S=\varnothing$, we have $u\notin \mathfrak{p}$; by primality,
        $$a \in \mathfrak{p} \ \text{or}\  b \in \mathfrak{p}.$$
        Hence $a/s$ or $b/t$ lies in $\mathfrak{p}^e$. 
        
        \item For any ideal $\mathfrak{q}\subset S^{-1}A$, we already proved $\mathfrak{q}^{ce}=\mathfrak{q}$.
        Now let $\mathfrak{p}\subset A$ be prime with $\mathfrak{p}\cap S=\varnothing$. Clearly $\mathfrak{p}^{ec} \supseteq \mathfrak{p}$.
        Take $a \in \mathfrak{p}^{ec}$, so $a/1 \in \mathfrak{p}^e$. Then $\exists s\in S$ such that $sa\in\mathfrak{p}$.
        But $s\notin\mathfrak{p}$, so $a\in\mathfrak{p}$ by primality. Thus $\mathfrak{p}^{ec}=\mathfrak{p}$.
    \end{enumerate}
\end{proof}
%==================== Localization Notes (Cleaned + Completed) ====================

\begin{proposition}
    There is a natural isomorphism of $S^{-1}A$-modules
    \[
    S^{-1}M \ \cong_{\mathrm{nat}}\ (S^{-1}A)\otimes_A M,
    \qquad 
    \frac{ax}{s}\ \leftrightarrows\ \frac{a}{s}\otimes x.
    \]
\end{proposition}

\begin{proof}
    Define a map
    \[
    \Phi:(S^{-1}A)\otimes_A M \longrightarrow S^{-1}M,
    \qquad 
    \Phi\!\left(\frac{a}{s}\otimes x\right)=\frac{ax}{s}.
    \]

    First observe that this map exists and is an $A$-module homomorphism (in fact, it is $S^{-1}A$-linear once it is well-defined).
    Indeed, it is induced by the bilinear map
    \[
    \beta:S^{-1}A\times M\to S^{-1}M,
    \qquad 
    \beta\!\left(\frac{a}{s},x\right)=\frac{ax}{s}.
    \]

    We check that $\beta$ is well-defined. If
    \[
    \frac{a}{s}=\frac{a'}{s'},
    \]
    then by definition there exists $u\in S$ such that
    \[
    u(s'a-sa')=0.
    \]
    Multiplying by $x\in M$ gives
    \[
    u(s'ax-sa'x)=0,
    \]
    hence in $S^{-1}M$ we have
    \[
    \frac{ax}{s}=\frac{a'x}{s'}.
    \]
    Therefore $\beta$ is well-defined, so it induces the $A$-linear map $\Phi$.

    Surjectivity is trivial: every element of $S^{-1}M$ has the form $x/s$, and
    \[
    \Phi\!\left(\frac{1}{s}\otimes x\right)=\frac{x}{s}.
    \]

    \medskip

    \textbf{Reduction to simple tensors.}
    A general element of $(S^{-1}A)\otimes_A M$ is a finite sum
    \[
    t=\frac{a_1}{s_1}\otimes x_1+\cdots+\frac{a_n}{s_n}\otimes x_n.
    \]
    Passing to a common denominator $s=s_1\cdots s_n$, we may rewrite
    \[
    t=\frac{b_1}{s}\otimes x_1+\cdots+\frac{b_n}{s}\otimes x_n
    =\frac{1}{s}\otimes\Big(\sum_i b_i x_i\Big).
    \]
    Thus every tensor may be represented in the form $\frac{1}{s}\otimes y$.

    \medskip

    \textbf{Alternate approach (constructing the inverse).}
    Define
    \[
    \Psi:S^{-1}M\to (S^{-1}A)\otimes_A M,
    \qquad 
    \Psi\!\left(\frac{x}{s}\right)=\frac{1}{s}\otimes x.
    \]

    We check that $\Psi$ is well-defined. Suppose
    \[
    \frac{x}{s}=\frac{x'}{s'}.
    \]
    Then there exists $u\in S$ such that
    \[
    u(s'x-sx')=0,
    \]
    i.e.
    \[
    us'x=usx'.
    \]
    Then in the tensor product,
    \[
    \frac{1}{s}\otimes x
    =\frac{u}{us}\otimes x
    =\frac{1}{us}\otimes ux
    =\frac{1}{us'}\otimes ux'
    =\frac{1}{s'}\otimes x'.
    \]
    Hence $\Psi$ is well-defined.

    Finally,
    \[
        (\Phi\circ\Psi)\!\left(\frac{x}{s}\right)
        =\Phi\!\left(\frac{1}{s}\otimes x\right)
        =\frac{x}{s}
    \]
    and
    \[
    (\Psi\circ\Phi)\!\left(\frac{a}{s}\otimes x\right)
    =\Psi\!\left(\frac{ax}{s}\right)
    =\frac{1}{s}\otimes ax
    =\frac{a}{s}\otimes x.
    \]
    Therefore $\Phi$ is an isomorphism, natural in $M$.
\end{proof}

\begin{remark}
    Why is it called \emph{localization}? Recall that there is an order-preserving bijection between primes:
    \[
    \Big\{\mathfrak p\subseteq A \ \big|\ \mathfrak p\cap S=\varnothing\Big\}
    \ \longleftrightarrow\
    \Big\{\mathfrak P\subseteq S^{-1}A\Big\},
    \qquad 
    \mathfrak p\mapsto S^{-1}\mathfrak p.
    \]

    In particular, if $S=A-\mathfrak p$, then
    \[
    A_{\mathfrak p}=S^{-1}A
    \]
    is a \emph{local ring}. Indeed, its unique maximal ideal is
    \[
    \mathfrak m_{\mathfrak p}
    =
    \left\{\frac{a}{s}\in A_{\mathfrak p}\ \middle|\ a\in\mathfrak p,\ s\notin\mathfrak p\right\}.
    \]
    Moreover,
    \[
    A_{\mathfrak p}-\mathfrak m_{\mathfrak p}
    =
    \left\{\frac{u}{s}\ \middle|\ u\notin\mathfrak p,\ s\notin\mathfrak p\right\}
    \]
    is exactly the set of units of $A_{\mathfrak p}$.

    \medskip

    A quick example: let $A=k[x,y,z]$ and $\mathfrak p=\langle x,y\rangle$. Then
    \[
    V(\mathfrak p)=\text{``the $z$-axis''}.
    \]
    Also $A/\mathfrak p\simeq k[z]$, via
    \[
    f(x,y,z)+\mathfrak p \ \longleftrightarrow\ f(0,0,z).
    \]
    Thus
    \[
    A_{\mathfrak p}
    =
    \left\{\frac{f(x,y,z)}{s(x,y,z)}\ \middle|\ s\notin\mathfrak p\right\}.
    \]
    The maximal ideal consists of fractions whose numerator vanishes on the $z$-axis:
    \[
    \mathfrak m_{\mathfrak p}
    =
    \left\{\frac{f(x,y,z)}{s(x,y,z)}\ \middle|\ f\in\mathfrak p\right\}.
    \]

    Moreover one checks that
    \[
    \frac{A_{\mathfrak p}}{\mathfrak m_{\mathfrak p}}
    \ \cong\
    \mathrm{Frac}(A/\mathfrak p)
    \ \cong\ k(z),
    \]
    the field of rational functions in $z$.
\end{remark}


Therefore we can define the \emph{localization functor}
\[
M\ \longmapsto\ S^{-1}M,
\]
which is secretly the functor
\[
M\ \longmapsto\ (S^{-1}A)\otimes_A M.
\]

Given a homomorphism $M\xrightarrow{f}N$, we define
\[
S^{-1}f:S^{-1}M\to S^{-1}N,
\qquad 
(S^{-1}f)\!\left(\frac{x}{s}\right)=\frac{f(x)}{s}.
\]

\begin{theorem}
    The localization functor is exact. Equivalently, $S^{-1}A$ is a flat $A$-algebra.
\end{theorem}

\begin{proof}
    Let
    \[
    M\xrightarrow{f}N\xrightarrow{g}P
    \]
    be exact at $N$, i.e. $\Im(f)=\ker(g)$.

    We claim that
    \[
    S^{-1}M\xrightarrow{S^{-1}f}S^{-1}N\xrightarrow{S^{-1}g}S^{-1}P
    \]
    is exact at $S^{-1}N$.

    Trivially $\Im(S^{-1}f)\subseteq\ker(S^{-1}g)$ since $gf=0$.

    Now let $y/s\in\ker(S^{-1}g)$. Then
    \[
    (S^{-1}g)\!\left(\frac{y}{s}\right)=\frac{g(y)}{s}=0
    \]
    in $S^{-1}P$, hence there exists $u\in S$ such that
    \[
    ug(y)=0,
    \quad\text{i.e.}\quad g(uy)=0.
    \]
    Thus $uy\in\ker(g)=\Im(f)$, so $uy=f(x)$ for some $x\in M$. Therefore
    \[
    \frac{y}{s}
    =
    \frac{uy}{us}
    =
    \frac{f(x)}{us}
    =
    (S^{-1}f)\!\left(\frac{x}{us}\right),
    \]
    proving $\ker(S^{-1}g)\subseteq\Im(S^{-1}f)$.
\end{proof}

\textbf{Immediate consequences of exactness.}

Let $M\xrightarrow{f}N$ be any homomorphism. Then localization preserves kernels, cokernels, and images:
\[
S^{-1}(\ker f)=\ker(S^{-1}f),
\qquad 
S^{-1}(\coker f)=\coker(S^{-1}f),
\qquad 
S^{-1}(\Im f)=\Im(S^{-1}f).
\]

If $N\subseteq M$, then
\[
S^{-1}(M/N)\ \cong\ (S^{-1}M)/(S^{-1}N),
\]
since localizing the short exact sequence
\[
0\to N\to M\to M/N\to 0
\]
remains exact.

Similarly,
\[
S^{-1}(M_1\oplus M_2)\ \cong\ S^{-1}M_1\oplus S^{-1}M_2.
\]

The direct sum maps fit into the standard diagram:
\[
\begin{tikzcd}
M_1 \arrow[r,"i_1"] & M_1\oplus M_2 \arrow[r,"p_2"] \arrow[l,bend left=30,"p_1"]
& M_2 \arrow[l,"i_2"']
\end{tikzcd}
\]
with relations
\[
p_1i_1=1,\quad p_2i_2=1,\quad p_1i_2=0,\quad p_2i_1=0,
\qquad 
i_1p_1+i_2p_2=1_{M_1\oplus M_2}.
\]

More interestingly, if $N_1,N_2\subseteq M$, then
\[
S^{-1}(N_1\cap N_2)=S^{-1}N_1\cap S^{-1}N_2,
\qquad 
S^{-1}(N_1+N_2)=S^{-1}N_1+S^{-1}N_2.
\]
Indeed we use the exact sequence
\[
0\to N_1\cap N_2\to N_1\oplus N_2
\xrightarrow{(y_1,y_2)\mapsto y_1-y_2}
N_1+N_2\to 0,
\]
where $z\mapsto(z,z)$ is the inclusion.

\begin{proposition}
    Let $M,N$ be $A$-modules. Then $\Hom_A(M,N)$ is an $A$-module, and there is a natural map
    \[
    S^{-1}\Hom_A(M,N)
    \xrightarrow{\ \varphi\ }
    \Hom_{S^{-1}A}(S^{-1}M,S^{-1}N),
    \]
    given by
    \[
    \frac{f}{s}\ \longmapsto\ \Big(\frac{x}{t}\mapsto \frac{f(x)}{st}\Big).
    \]
    Moreover:
    \begin{itemize}
        \item If $M$ is finitely generated, then $\varphi$ is injective.
        \item If $M$ is finitely presented, then $\varphi$ is bijective.
    \end{itemize}
    (Sketch: finite generation controls kernels, finite presentation controls surjectivity.)
\end{proposition}

\subsection*{Local vs global properties}

We know: for $x\in M$, when is $x/1=0$ in $S^{-1}M$? Precisely when
\[
\exists s\in S\text{ such that }sx=0,
\qquad\text{i.e.}\qquad \Ann(x)\cap S\neq\varnothing.
\]

Fix $x\in M$. Which primes $\mathfrak p$ satisfy $x/1\neq 0$ in $M_{\mathfrak p}$?
These are exactly
\[
\{\mathfrak p\mid \Ann(x)\subseteq\mathfrak p\}.
\]

\begin{example}
    Let $A=\Z$ and $M=\Z/20\Z$. Then
    \[
    \Ann([1])=20\Z\subseteq 2\Z,5\Z,
    \qquad 
    \Ann([4])=5\Z.
    \]
    Hence
    \[
    M_{(2)}\cong \Z/4\Z,
    \qquad 
    M_{(5)}\cong \Z/5\Z.
    \]
\end{example}

\begin{example}
    Let $A=\C[x,y]$ and maximal ideals
    \[
    \mathfrak m_{a,b}=\langle x-a,\ y-b\rangle.
    \]
    If
    \[
    M=A/\langle x^2+y^2-1\rangle,
    \]
    then $[1]$ lives at $\mathfrak m_{a,b}$ iff
    \[
    x^2+y^2-1\in\mathfrak m_{a,b}
    \iff a^2+b^2=1.
    \]
\end{example}

\begin{example}
    Consider $[x]\in k[x,y]/\langle x,y\rangle$ as a $k[x,y]$-module element. Then
    \[
    \Ann([x])=\langle y\rangle,
    \]
    so $[x]$ lives exactly along the $x$-axis.
\end{example}

\paragraph{Local-global principles}

\begin{proposition}
    Let $M$ be an $A$-module. Then the following are equivalent:
    \begin{itemize}
        \item $M=0$.
        \item For all primes $\mathfrak p$, $M_{\mathfrak p}=0$.
        \item For all maximal ideals $\mathfrak m$, $M_{\mathfrak m}=0$.
    \end{itemize}

    Equivalently, for $x\in M$, the following are equivalent:
    \begin{itemize}
        \item $x=0$.
        \item For all primes $\mathfrak p$, $x/1=0$ in $M_{\mathfrak p}$.
        \item For all maximal ideals $\mathfrak m$, $x/1=0$ in $M_{\mathfrak m}$.
    \end{itemize}
\end{proposition}

\begin{proof}
    $(1)\implies(2)\implies(3)$ is trivial.

    For $(3)\implies(1)$: assume $M_{\mathfrak m}=0$ for all maximal ideals.
    Let $x\in M$ with $x\neq 0$. Then $\Ann(x)$ is a proper ideal, hence contained in some maximal ideal $\mathfrak m$:
    \[
    \Ann(x)\subseteq\mathfrak m.
    \]
    But $x/1=0$ in $M_{\mathfrak m}$ means there exists $s\notin\mathfrak m$ such that $sx=0$.
    Thus $s\in\Ann(x)\subseteq\mathfrak m$, contradiction.
    Therefore $x=0$, so $M=0$.
\end{proof}

\textbf{Consequences.}

\begin{proposition}
    A homomorphism $f:M\to N$ is injective (resp.\ surjective, resp.\ bijective) iff
    for all primes $\mathfrak p$ (equivalently, all maximals $\mathfrak m$),
    \[
    f_{\mathfrak p}:M_{\mathfrak p}\to N_{\mathfrak p}
    \]
    is injective (resp.\ surjective, resp.\ bijective).
\end{proposition}

\begin{proof}
    Apply localization to the exact sequences involving $\ker f$ and $\coker f$.
\end{proof}

\begin{proposition}
    If $N\subseteq M$, then
    \[
    N=M
    \iff
    N_{\mathfrak p}=M_{\mathfrak p}\ \forall\mathfrak p
    \iff
    N_{\mathfrak m}=M_{\mathfrak m}\ \forall\mathfrak m.
    \]
\end{proposition}

\begin{proof}
    Apply the previous proposition to $M/N$.
\end{proof}

\begin{proposition}
    If $x\in M$ satisfies $x/1=0$ in $M_{\mathfrak m}$, then the image of $x$ in
    \[
    M/\mathfrak m M \ \cong\ (A/\mathfrak m)\otimes_A M
    \]
    is zero.
\end{proposition}

\begin{proof}
    If $x/1=0$ in $M_{\mathfrak m}$, then $\exists s\notin\mathfrak m$ with $sx=0$.
    In $M/\mathfrak m M$, we have $s[x]=0$, but also $\mathfrak m[x]=0$.
    Since $s\notin\mathfrak m$, the ideal $\mathfrak m+\langle s\rangle=A$,
    hence $[x]=0$.
\end{proof}

\begin{theorem}
    Suppose $M$ is a finitely generated $A$-module. Then
    \[
    M=0
    \iff
    \forall\mathfrak m,\ M/\mathfrak m M=0.
    \]
\end{theorem}

\begin{proof}
    Assume $M/\mathfrak m M=0$ for all maximal ideals $\mathfrak m$.

    Localizing at $\mathfrak m$, we get
    \[
    M_{\mathfrak m}/\mathfrak m_{\mathfrak m}M_{\mathfrak m}
    \ \cong\
    (M/\mathfrak m M)_{\mathfrak m}
    =0.
    \]
    Since $M_{\mathfrak m}$ is finitely generated over the local ring $A_{\mathfrak m}$,
    Nakayama's lemma implies
    \[
    M_{\mathfrak m}=0
    \quad \forall\mathfrak m.
    \]
    By the local-global principle, $M=0$.
\end{proof}

\begin{remark}
    Finite generation is essential. For example, if $A=\Z$ and $M=\Q/\Z$, then
    multiplication by $n$ is surjective for all $n$, hence
    \[
    (\Q/\Z)/p(\Q/\Z)=0\quad \forall p,
    \]
    but $\Q/\Z\neq 0$.
\end{remark}
\end{document}